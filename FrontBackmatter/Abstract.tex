%*******************************************************
% Abstract ( 350 words max ! )
%*******************************************************
\begingroup
\let\clearpage\relax
\let\cleardoublepage\relax
\let\cleardoublepage\relax
\chapter*{Abstract}
Essential microscopic aspects of activated transport in liquids, which
precedes the glass transition, have evaded explanation for
decades. These poorly understood aspects include: the molecular
underpinning of the excess, configurational entropy; the transition
state configurations for the activated transport; the chemical origin
of the fragile vs. strong liquid behavior; and many others. This
dissertation puts forth a radically novel way to address these open
questions, in which liquids near their glass transition are viewed as
structurally degenerate assemblies of strongly interacting, local
sources of frozen-in stress. The thermodynamics and activated barriers
for rearrangement of this stress field have been mapped onto a
Heisenberg model with six-dimensional spins. A meanfield analysis of
the spin model has shown glasses can be viewed as frozen-in patterns
of shear stress and/or uniform compression/dilation, the two extremes
corresponding to the strong and fragile behaviors. A self-consistent
elasticity theory of aperiodic, metastable solids emerges in the
present analysis; it supersedes the traditional elasticity theory,
which fails to self-consistently account for the structural degeneracy
stemming from the inherent mismatch between cohesive forces and steric
repulsion.  The observable elastic constants self-consistently emerge
in the present theory similarly to how the dielectric susceptibility
is determined by the properties of molecular dipoles. First
simulations of the spin model have been carried out. In addition to
direct observations of transition states for the activated transport,
several key features of the glass transition are yielded by the spin
model, including a strongly non-exponential, non-Arrhenius character
of the relaxations and its correlation with the Poisson ratio of the
substance.
\endgroup			
\vfill
