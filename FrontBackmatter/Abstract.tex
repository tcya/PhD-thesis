%*******************************************************
% Abstract ( 350 words max ! )
%*******************************************************
\begingroup
\let\clearpage\relax
\let\cleardoublepage\relax
\let\cleardoublepage\relax
\chapter*{Abstract}
% Essential microscopic aspects of activated transport in liquids, which
% precedes the glass transition, have evaded explanation for
% decades. These poorly understood aspects include: the molecular
% underpinning of the excess, configurational entropy; the transition
% state configurations for the activated transport; the chemical origin
% of the fragile vs. strong liquid behavior; and many others. This
% dissertation puts forth a radically novel way to address these open
% questions, in which liquids near their glass transition are viewed as
% structurally degenerate assemblies of strongly interacting, local
% sources of frozen-in stress. The thermodynamics and activated barriers
% for rearrangement of this stress field have been mapped onto a
% Heisenberg model with six-dimensional spins. A meanfield analysis of
% the spin model has shown glasses can be viewed as frozen-in patterns
% of shear stress and/or uniform compression/dilation, the two extremes
% corresponding to the strong and fragile behaviors. A self-consistent
% elasticity theory of aperiodic, metastable solids emerges in the
% present analysis; it supersedes the traditional elasticity theory,
% which fails to self-consistently account for the structural degeneracy
% stemming from the inherent mismatch between cohesive forces and steric
% repulsion.  The observable elastic constants self-consistently emerge
% in the present theory similarly to how the dielectric susceptibility
% is determined by the properties of molecular dipoles. First
% simulations of the spin model have been carried out. In addition to
% direct observations of transition states for the activated transport,
% several key features of the glass transition are yielded by the spin
% model, including a strongly non-exponential, non-Arrhenius character
% of the relaxations and its correlation with the Poisson ratio of the
% substance.
We present a new, fully {\em ab initio} approach for computing intramolecular charge and
energy transfer rates.  Using a time-convolutionless master equation approach, parameterized
with couplings obtained from an accurate quantum chemical approach, we  benchmark the approach
against  experimental results and predictions from Marcus theory for triplet energy transfer for a
series of donor-bridge-acceptor systems.
An important component of our analysis is the use of a projection operator scheme
that parses out specific internal nuclear motions that accompany
the electronic transition.
Using an iterative Lanczos method, we  concentrate the coupling between the electronic and nuclear
degrees of freedom into a small number of reduced harmonic modes.
We find that  using only a single reduced mode--termed the ``primary mode'' or ``Lanczos modes'',
 one obtains an accurate evaluation of the golden-rule rate constant and
insight into the nuclear motions responsible for coupling the initial and final electronic states.

In particular, the irreducible representation of the primary mode reveals
hidden details of the dynamics. For the cases considered here, the primary modes belong to totally symmetric irreducible representations  of
the donor and acceptor moieties. Upon investigating the molecular geometry changes following  the transition,   we propose that the electronic transition process can be
broken into two steps, in the agreement of Born-Oppenheimer approximation:  a fast excitation transfer occurs, facilitated by the ``primary Lanczos mode'' (PLM),
followed by slow nuclear relaxation on the final electronic diabatic surface.

We apply the method to a larger, ``star'' molecule, that has been experimentally shown  that its exciton transfer pathway can be radically modified by mode-specific infrared excitation of its vibrational mode. The primary mode and rate constants we obtain generally agree with the experiments.
\endgroup
\vfill
