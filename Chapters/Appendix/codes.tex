%!TEX root = /Users/tcya/Dropbox/PhDthesis/thesis.tex
\chapter{Sample Codes}
The codes below are for the computation of rate constants and projected modes, written in Mathematica.
\\\large{Helper functions}
\begin{lstlisting}
ReadRotM[filename_]: = Module[{stream, RotMat},
stream = OpenRead[filename];
RotMat = Transpose[Partition[ToExpression[Last[Transpose[StringSplit[FindList[filename, "diabatic RotMatrix"]]]]], 2]];
Close[stream];
Return[RotMat]]

ReadEne[filename_]: = Module[{stream, e1, e2},
stream = OpenRead[filename];
e1 = ToExpression[Last[StringSplit[Find[stream, "showmatrix adiabatH(0, 0)"]]]]*27.211396132;
e2 = ToExpression[Last[StringSplit[Find[stream, "showmatrix adiabatH(1, 1)"]]]]*27.211396132(*in eV*);
Close[stream];
Return[{e1, e2}](* in hartree *)]

ReadNormalModes[filename_](* for QChem*):= Module[{stream, NAtom, NMode, NBlock, Freq, ForceConst, RedMass, IRActive, IRIntens, RamanActive, NormalModeCoord, IAn, AtomicMass, tmp},
  Freq = {};
  ForceConst = {};
  RedMass = {};
  IRActive = {};
  IRIntens = {};
  RamanActive = {};
  NormalModeCoord = {};
  IAn = {};
  AtomicMass = {};

  stream = OpenRead[filename];
  NMode = ToExpression[Last[Last[StringSplit[FindList[stream, "Mode"]]]]];
  NBlock = If[IntegerQ[#], #, Ceiling[#]]&[NMode/3];
  NAtom = (NMode + 6)/3;
  Close[stream];

  stream = OpenRead[filename];
  Find[stream, "Raman Active"];
  Read[stream];
  tmp = Table[First[StringSplit[Read[stream, String]]], {NAtom}];
  AtomicMass = QuantityMagnitude[ElementData[#, "AtomicWeight"] & /@ tmp];
  IAn = ElementData[#, "AtomicNumber"] & /@ tmp;
  Close[stream];

  stream = OpenRead[filename];
  Do[
   Find[stream, "Mode"];
   AppendTo[Freq, ToExpression[Drop[StringSplit[Read[stream, String]], 1]]];
   AppendTo[ForceConst, ToExpression[Drop[StringSplit[Read[stream, String]], 2]]];
   AppendTo[RedMass, ToExpression[Drop[StringSplit[Read[stream, String]], 2]]];
   AppendTo[IRActive, Drop[StringSplit[Read[stream, String]], 2]];
   AppendTo[IRIntens, ToExpression[Drop[StringSplit[Read[stream, String]], 2]]];
   AppendTo[RamanActive, Drop[StringSplit[Read[stream, String]], 2]];
   Read[stream];
   NormalModeCoord = Join[NormalModeCoord, ToExpression[Transpose[Table[Partition[Rest[StringSplit[Read[stream, String]]], 3], {NAtom}]]]];
   , {i, 1, NBlock}];
  Close[stream];

  {Freq, ForceConst, RedMass, IRActive, IRIntens, RamanActive} = Flatten /@ {Freq, ForceConst, RedMass, IRActive, IRIntens, RamanActive};
  Return[{NAtom, NMode, AtomicMass, Freq, RedMass, NormalModeCoord}]]

ReadGradient[filename_, NAtom_] := Module[{stream, grad, NMode, NBlock, tmp},
  grad = {};
  NBlock = If[IntegerQ[#], #, Ceiling[#]]&[NAtom/6];

  stream = OpenRead[filename];
  Find[stream, "Gradient of the state energy (including CIS Excitation Energy)"];
   Do[
    Read[stream];
    grad = Join[grad, Transpose[Table[ToExpression[Rest[StringSplit[Read[stream, String]]]], {3}]]];
    , {i, 1, NBlock}];
  Return[grad]]
\end{lstlisting}
\begin{lstlisting}
A_$\otimes$B_:= Outer[Times, A, B]
(* Project1by1 sorts phono modes into a 1 by 1 hierarchical form. Only works for one mode case, for general 3 by 3 case it should be modified. *)
Project1by1[gg_, Hess_]: = Module[
{S, P, Q, PHP, QHQ, L, $\omega$c, $\omega$ck, $\omega$b, Mc, Mck, Mb, np, ns, Sinv, G, p, k, g},
(* g should be 1-D, namely, ns = 1 *)
{ns, np} = Dimensions[gg];
g = gg;
$\omega$c = {};
Mc = {};
p = ConstantArray[0, {np, np}];

Do[{
S = Table[g[[i]].g[[j]], {i, ns}, {j, ns}];
Sinv = Inverse[S] ;
(* define projection operators *)
P = Sum[Sinv[[i, j]]  g[[i]]\[CircleTimes]g[[j]], {i, ns}, {j, ns}] ;
p = p+P; (*$P = P_1 + P_2 + ...$*)
Q = IdentityMatrix[np]-p;
PHP = P.Hess.P ;
QHQ = Q.Hess.Q ;
{$\omega$ck, Mck} = Chop[Transpose[Take[Sort[Transpose[Eigensystem[PHP]]], -ns]], 10^-8];
{$\omega$b, Mb} = Chop[Transpose[Take[Sort[Transpose[Eigensystem[QHQ]]], ns+k-1-np]], 10^-8];
$\omega$c = Chop[Join[$\omega$c, $\omega$ck], 10^-8];
Mc = Chop[Join[Mc, Mck], 10^-8];
$\omega$all= Join[$\omega$c, $\omega$b]//Chop;
M = Join[Mc, Mb]//Chop ;
g = Chop[{Join[ConstantArray[0, k], ( M.Hess.Transpose[M])[[k]][[k+1;;np]]].M}, 10^-8]; (* Define nth mode = (0, 0, 0, ..., $\gamma_{k+1}$, $\gamma_{k+2}$, ..., $\gamma_{k+N}$) *)
Gnew = M.gg[[1]];
}, {k, 1, np-1}];
](* The kth step gives k*k tridiagonal submatrix on lefttop, diagonal N-k on rightbottom and only the kth mode is coupled to the N-k bath modes *)
\end{lstlisting}
\begin{lstlisting}
NMode[n_, hess_, M_, g_]: = Module[{h, gn, Mnm, $\omega$prime, Gn},
h = Take[(M.hess.Transpose[M]), {1, n}, {1, n}];
{$\omega$prime, Mn} = Transpose[Sort[Transpose[Eigensystem[h]]]];
gn = Join[Take[g, n], ConstantArray[0, n-Length[Take[g, n]]]];
Gn = Mn.gn;
Return[{$\sqrt{\omega prime}$, Gn}]]
\end{lstlisting}
\large{Initialization}
\begin{itemize}[leftmargin = *]
\item Load Core routines
\begin{lstlisting}
SetDirectory[NotebookDirectory[]];
Needs["PhysicalConstants`"];  (* this package is outdated...but highly useful! *)
\end{lstlisting}

\item Conversion factors
\begin{lstlisting}
hartree = 27.211396132 ElectronVolt(* eV *);
Bohr = 0.5291772108 Angstrom;
$\hbar$ = hbar = PlanckConstantReduced;
\end{lstlisting}

\item Read in files
\begin{lstlisting}
name = {"c13ea", "c13ee", "c14ea", "c14ee", "d26ae", "d26ea", "d26ee", "d27ae", "d27ea", "d27ee", "mmm"};
path = "qchemout/"<>#<>".rtf"&/@name;
\end{lstlisting}
\begin{lstlisting}
kkk = 4;  (* in this case we are looking at c14ee *)
RotMat = ReadRotM[path[[kkk]]];
Ene = ReadEne[path[[kkk]]];
{NAtom, dof, AtomicMass, AngFreq, RedMass, NormMode} = ReadNormalModes[path[[kkk]]];
Grads = ReadGradient[path[[kkk]], NAtom];
$\omega$ = FreqeV = Table[Convert[2Pi SpeedOfLight AngFreq[[i]]Centimeter^-1 PlanckConstantReduced, ElectronVolt]/ElectronVolt, {i, 1, dof}];
Freq = Table[Convert[2Pi SpeedOfLight AngFreq[[i]]Centimeter^-1 , Second^-1], {i, 1, dof}];
NormModeInMWC = Table[Table[NormMode[[m, i, j]]$\sqrt{AtomicMass[[i]]}/\sqrt{RedMass[[m]]}$//Simplify//Chop, {i, NAtom}, {j, 3}], {m, 1, dof}];
Force = Table[Sum[NormModeInMWC[[m]][[i, j]]*Grads[[i, j]]/$\sqrt{AtomicMass[[i]]}$//Simplify//Chop, {i, 1, NAtom}, {j, 1, 3}], {m, 1, dof}];
GGG = CouplingConstant = Table[Convert[Force[[m]]hartree/Bohr$\sqrt{PlanckConstantReduced/(2 Freq[[m]]AMU)}$, ElectronVolt]/.ElectronVolt->1//Simplify//Chop, {m, 1, dof}];
G = G0 = Table[({
 {RotMat[[1, 2]]^2, RotMat[[2, 2]]RotMat[[1, 2]]},
 {RotMat[[2, 2]]RotMat[[1, 2]], RotMat[[2, 2]]^2}
})*CouplingConstant[[m]]//Simplify//Chop, {m, 1, dof}];
\end{lstlisting}
\large{Mode Analysis Routines}
\item Analyze n projected modes
\begin{lstlisting}
ModeAnalysis[n_]: = Module[{$\omega$prime},
Clear[imax];
imax = n;
{$\omega$prime, g} = NMode[n, hmat, M, Gnew];
Clear[$\omega$, G];
$\omega$ = Take[$\omega$prime, imax];
G = Table[({
 {RotMat[[1, 2]]^2, RotMat[[2, 2]]RotMat[[1, 2]]},
 {RotMat[[2, 2]]RotMat[[1, 2]], RotMat[[2, 2]]^2}
})*g[[m]]//Simplify//Chop, {m, 1, imax}];

v12tab[kkk][imax] = Table[{t, V[1, 2, t]}, {t, 0, nstep*tstep, tstep}];
v21tab[kkk][imax] = Table[{t, V[2, 1, t]}, {t, 0, nstep*tstep, tstep}];
v12[kkk][imax]= Interpolation[v12tab[kkk][imax]];
v21[kkk][imax]= Interpolation[v21tab[kkk][imax]];
b12tab[kkk][imax] = Table[{t, Bd[1, 2, t]}, {t, 0, nstep*tstep, tstep}];
b21tab[kkk][imax] = Table[{t, Bd[2, 1, t]}, {t, 0, nstep*tstep, tstep}];
b12d[kkk][imax]= Interpolation[b12tab[kkk][imax]];
b21d[kkk][imax]= Interpolation[b21tab[kkk][imax]];]
\end{lstlisting}
\item Calculation of Correlation Functions
\begin{lstlisting}
nstep = 3000;
tstep = 0.14;
imin = 1;
imax = dof;
Temperature = 298 Kelvin;

$\hbar$ = hbar= 6.582122*10^(-4)(* eV/ps^-1 =Convert[PlanckConstantReduced /10^-12 /Second, ElectronVolt]*);
$\beta$ = Convert[1/(Temperature BoltzmannConstant), ElectronVolt^-1]/.ElectronVolt->1;
Clear[$\epsilon$];
$\epsilon$[n_]: = Ene[[n]]-Sum[(G[[i, n, n]]^2/$\omega$[[i]]), {i, imin, imax}];
nu[i_]: = 1/(Exp[$\beta$*$\omega$[[i]]]-1);

$\Delta$[n_, m_, i_]: = (G[[i, n, n]]-G[[i, m, m]])/$\omega$[[i]];
$\Omega$[n_, m_, i_]: = (G[[i, n, n]]+G[[i, m, m]])/$\omega$[[i]];
f[n_, m_, $\tau$_]: = Exp[-2*Sum[(nu[i]+0.5)*($\Delta$[n, m, i])^2*(1-Cos[$\omega$[[i]]*$\tau$]), {i, imin, imax}]];
q[n_, m_, $\tau$_]: = Exp[I*Sum[($\Delta$[n, m, i])^2*Sin[$\omega$[[i]]*$\tau$], {i, imin, imax}]];
o[n_, m_, $\tau$_]: = (Sum[G[[i, n, m]]*($\Delta$[n, m, i]*(nu[i]+1)Exp[I*$\omega$[[i]]*$\tau$]-$\Delta$[n, m, i]*nu[i]*Exp[-I*$\omega$[[i]]*$\tau$]+$\Omega$[n, m, i]), {i, imin, imax}])^2+Sum[G[[i, n, m]]*G[[i, m, n]]*((nu[i]+1)Exp[I*$\omega$[[i]]*$\tau$]+nu[i]Exp[-I*$\omega$[[i]]*$\tau$]), {i, imin, imax}];
V[m_, n_, $\tau$_]: = 2*f[n, m, $\tau$]*Re[Exp[-I*($\epsilon$[n]-$\epsilon$[m])*$\tau$]*q[n, m, $\tau$]*o[n, m, $\tau$]];
Bd[m_, n_, $\tau$_]: = f[n, m, $\tau$]*q[n, m, $\tau$]*o[n, m, $\tau$];
\end{lstlisting}
\large{Get G and [Omega]}
\begin{lstlisting}
hess = FreqeV^2;
hmat = DiagonalMatrix[hess];
Project1by1[{CouplingConstant}, hmat]; (*This  gives a fully sorted tridiagonal matrix. You may encouter singular matrix error for certain molecules, c14ee for example, because some couplings are zero. Manually abort it and continue.*)
\end{lstlisting}
\begin{lstlisting}
ModeAnalysis[dof](*Exact calculation*)
\end{lstlisting}
\begin{lstlisting}
ModeAnalysis[1](*Do however many modes you like.*)
ModeAnalysis[4]
ModeAnalysis[7]
Do[ModeAnalysis[i], {i, 5, 20, 5}]
\end{lstlisting}
\large{Rate}
\begin{lstlisting}
$\phi$12 = Table[NIntegrate[v12[kkk][dof][$\tau$], {$\tau$, i*tstep, (i+1)*tstep}], {i, 0, nstep-1}];
$\phi$21 = Table[NIntegrate[v21[kkk][dof][$\tau$], {$\tau$, i*tstep, (i+1)*tstep}], {i, 0, nstep-1}];
Ww12 = {{0, 0}};
Ww21 = {{0, 0}};
$\omega$$\omega$12 = 0;
$\omega$$\omega$21 = 0;
Do[$\omega$$\omega$12p = $\omega$$\omega$12+$\phi$12[[i]];
Ww12 = Append[Ww12, {i*tstep*$\hbar$, $\omega$$\omega$12p}];
$\omega$$\omega$12 = $\omega$$\omega$12p, {i, nstep}];
Do[$\omega$$\omega$21p = $\omega$$\omega$21+$\phi$21[[i]];
Ww21 = Append[Ww21, {i*tstep*$\hbar$, $\omega$$\omega$21p}];
$\omega$$\omega$21 = $\omega$$\omega$21p, {i, nstep}];
W12t= Interpolation[Ww12];
W21t= Interpolation[Ww21];
vm12 = Mean@Take[Table[Ww12[[i]][[2]], {i, 1, nstep}], -30];
vm21 = Mean@Take[Table[Ww21[[i]][[2]], {i, 1, nstep}], -100];
W12 = Function[Piecewise[{{W12t[#], 0< = #< = nstep*tstep}, {vm12 , #>nstep*tstep*$\hbar$}}]];
W21 = Function[Piecewise[{{W21t[#], 0< = #< = nstep*tstep}, {vm21 , #>nstep*tstep*$\hbar$}}]];
Print["Golden rule rate =", vm21/$\hbar$*10^12, "$s^{-1}$"];
\end{lstlisting}
\large{Correlation function}
\begin{lstlisting}
 Plot[{
  Re[b21d[kkk][dof][.001 t/$\hbar$]],
  Re[b21d[kkk][1][.001 t/$\hbar$]], Re[b21d[kkk][4][.001 t/$\hbar$]],
  Re[b21d[kkk][7][.001 t/$\hbar$]]},
 {t, 0, 10},
 PlotRange -> All,
 AspectRatio -> 1,
 PlotStyle -> {{Black, Thick, Dashing[0.01]}, Blue, Green, Red, Purple},
 LabelStyle -> 16,
 AxesLabel -> {"t/fs", "Re($C_{12}(t)$))"},
 PlotLegends -> Placed[{"Exact", "1 Mode", "4 Modes", "7 Modes", "7 Modes"}, Right],
 Epilog -> {Style[Text[name[[kkk]], {9, -2.5*10^-7}], 24]}]
\end{lstlisting}

\end{itemize}
